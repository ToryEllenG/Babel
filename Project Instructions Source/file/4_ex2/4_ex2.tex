\documentclass[main.tex]{subfiles}
\begin{document}
\subsection{Exercise 2: Setting Up a Mesh Network \& Convergence Times  }
\newlist{questions}{enumerate}{1}
\setlist[questions,1]{label={$\circ$ Q\arabic* -}}
\begin{itemize}


\subsubsection{Step 0: Preparation}
\begin{enumerate}[noitemsep,label=$\bullet$,leftmargin=20mm,labelsep=0.5cm]
\item To begin this exercise create two more linked clones of the babel ubuntu Server VMs and 1 more linked clone of a client VM of your choosing(we used Ubuntu Desktop VMS).
\item In the virtual network editor in VMWare we will need:
\item 4 host-only vmnets w/o DHCP for VM1(Babel1).
\item 3 host-only vmnets w/o DHCP for VM2(Babel2).
\item 3 host-only vmnets w/o DHCP for VM3(Babel3).
\item 4 host-only vmnets w/o DHCP for VM4(Babel4).
\item You will also need 2 additional vmnets host-only w/o dhcp for the 2 linked Ubuntu Desktop VMs (or whatever you choose to be your clients).
\item You will need a total of 16 vmnets
\item On VM1 (Babel1) add 4 network adapters and select any of the custom vmnets so that they are different for each interface.
\item Repeat the same procedure for VM4(Babel4).
\item On VM2(Babel2) add 3 network adapters and assign different vmnets than the ones selected for VM1 and VM4.
\item Repeat the Same procedures for VM3(Babel3).
\item Connect all the links and assign static IPs to all of the interfaces as shown in Figure \ref{fig:ex2Net}.
\item Now babeld should've been installed on the VM you created your linked clones from, verify that it's installed using the command \texttt{babeld --version} on all of your babel VMs.
\item Make sure your client VMs can pinged there respective gateways (which should have been added when assigning static IPs to them)
\end{enumerate}
\ExecuteMetaData[file/4_ex2/4_ex2_CBQMs]{mtag1}

\subsubsection{Step 1: Convergence Time in Comparison to RIP } 
\begin{enumerate}[noitemsep,label=$\bullet$,leftmargin=20mm,labelsep=0.5cm]
\item Next, we are going to start babel on the 4 VMs that are acting as routers.
\item  On Babel1, from the terminal launch the command \texttt{sudo babeld -d1 -C 'redistribute metric 128' eth0 eth1 eth2}.  You should start to see some debug messages after a few seconds.
\item On Babel2, start babel with the command \texttt{sudo babeld -d1 eth0 eth1 eth2}
\item Repeat the same command as Babel1 on Babel4.
\item Repeat the same command as Babel2 on Babel3.
\item Next, we are going to test the convergence time of bringing a link down
\item Start pinging UbuntuHost2 from UbuntuHost1.(should be successful at this point)
\item Next, bring down interface eth2 on Babel4.
\item Watch the ping on Host1 and record the time from when it pauses to when it begins to to sending request and receiving replys again.
\end{enumerate}
\ExecuteMetaData[file/4_ex2/4_ex2_CBQMs]{mytag1}
\ExecuteMetaData[file/4_ex2/4_ex2_CBQMs]{mytag2}
\begin{enumerate}[noitemsep,label=$\bullet$,leftmargin=20mm,labelsep=0.5cm]
\item Next, launch wireshark on the link between Babel1 and Babel4 and the link between Babel1 and Babel3 (or which ever link is currently capturing the ICMP packets)
\item Bring interface eth2 back up on Babel2 and watch the Wireshark capture on the link between Babel1 and Babel2.
\end{enumerate}
\ExecuteMetaData[file/4_ex2/4_ex2_CBQMs]{mytag3}


\subsubsection{Step 2: Routing Tables Affecting Traceroutes}
\begin{enumerate}[noitemsep,label=$\bullet$,leftmargin=20mm,labelsep=0.5cm]
\item Verify that traceroute is installed on Host1.
\item  Do a trace route from Host1 to Host2 and record the path.
\item Launch wireshark on the Link between Babel1 and Babel4, Babel1 and Babel3, Babel1 and Babel2, Babel2 and Babel4.
\item Bring interface eth2 down on Babel4 again.
\item Run traceroute again (you may need to keep trying until the convergence is complete.
\end{enumerate}
\ExecuteMetaData[file/4_ex2/4_ex2_CBQMs]{mytag4}
\ExecuteMetaData[file/4_ex2/4_ex2_CBQMs]{mytag5}



\subsubsection{Step 3: Compare and Contrasting RIP and Babel}
\begin{enumerate}[noitemsep,label=$\bullet$,leftmargin=20mm,labelsep=0.5cm]
\item Take a look back at Lab 2 particularly at how static routes were distributed in RIP.

\ExecuteMetaData[file/4_ex2/4_ex2_CBQMs]{mytag6}
\ExecuteMetaData[file/4_ex2/4_ex2_CBQMs]{mytag7}

\item Another major difference between RIPv2 and Babel is their operating environment. RIP runs on Cisco IOS and we have Babel running on Ubuntu Servers.

\item Babel works with both IPv4 and IPv6 by default. In RIP, you need to make separate configurations for IPv6 compatability.

\ExecuteMetaData[file/4_ex2/4_ex2_CBQMs]{mytag8}

\end{enumerate}
\end{itemize}
\end{document}
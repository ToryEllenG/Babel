\documentclass[main.tex]{subfiles}
\begin{document}
\hfill \break
The babel routing protocol is a distance-vector protocol in which was designed with the RIP protocol, with some additional features. It is widely known as "Speedy RIP", and is primarily a loop-avoiding protocol that was originally designed for wireless ad-hoc networks. This being said, babel is still viable and stable in wired networks, as well as hybrid networks consisting of wireless and wired nodes. 

Babel limits the frequency and duration of routing paths when a path is lost or dropped. Based on this capability, Babel is very efficient at reconverging to another path in such a scenario. The is done using a technique called DSDV, or Destination Sequenced Distance-Vector" routing, based on the Bellman-Ford algorithm.

Being a "double-stack" routing protocol, Babel is supported in both IPv4 and IPv6 networks. In addition to this, Babel can automatically detect wireless and wired interfaces and adjust accordingly.

In this lab / Semester Project, We covered 5 main exercises. These being:

\begin{enumerate}
    \item Preliminary Network Topology and Babel installation of Ubuntu Server VMs using GNS3
    \item Settings up a Mesh Network Architecture and Convergence Time Investigation
    \item Configuring OpenWRT and Setting up a Babel Network on Physical Routers
    \item Creation of a Babel Enabled Network Hybrid Network Architecture.
    \item Security Issues with Babel
\end{enumerate}



\end{document}
